\documentclass{article}
% Language setting
% Replace `english' with e.g. `spanish' to change the document language
%\usepackage[english]{babel}
% Set page size and margins
% Replace `letterpaper' with `a4paper' for UK/EU standard size
%\usepackage[letterpaper,top=2cm,bottom=2cm,left=3cm,right=3cm,marginparwidth=1.75cm]{geometry}
% Useful packages
%\usepackage{amsmath}
\usepackage{graphicx}
%% nessesary
\usepackage[colorlinks=true, allcolors=blue]{hyperref}
\usepackage[rflt]{floatflt}
\usepackage[T1]{fontenc}
\usepackage{textcomp}
\usepackage[utf8x]{inputenc}
\pagestyle{empty}

\begin{document}
\openup.5em

\begin{floatingfigure}{2.9in}
%\resizebox{2.5in}{!}{\input{r4}}
\scalebox{.5}{\input{r4}}
\end{floatingfigure}

\noindent The figure to the right provides an illustration of L’H\^o\-pi\-tal’s Rule.  Recall
that this rule can be applied when taking the limit as $x\longrightarrow x_a$
of a ratio of two functions where the ratio approaches the indeterminate form
$\frac{0}{0}$; in the case where both functions are differentiable at $x_a$,
the ratio approaches the ratio of their derivatives. In the case illustrated both
$\sin(x)$ and $x \longrightarrow0$ as we approach the origin, but the ratio of
their derivatives, $\frac{\cos(x)}{1} \longrightarrow 1$. L’H\^opital’s Rule
also applies in the case of the indeterminate form $\frac{\infty}{\infty}$.
%\usepackage{gnuplot-lua-tikz}
\end{document}

%%\documentclass{article}
%%% Language setting
%%% Replace `english' with e.g. `spanish' to change the document language
%%\usepackage[english]{babel}
%%% Set page size and margins
%%% Replace `letterpaper' with `a4paper' for UK/EU standard size
%%\usepackage[letterpaper,top=2cm,bottom=2cm,left=3cm,right=3cm,marginparwidth=1.75cm]{geometry}
%%% Useful packages
%%\usepackage{amsmath}
%%\usepackage{graphicx}
%%\usepackage[colorlinks=true, allcolors=blue]{hyperref}
%%\usepackage[rflt]{floatflt}
%%\usepackage[T1]{fontenc}
%%\usepackage{textcomp}
%%\usepackage[utf8x]{inputenc}
%%\pagestyle{empty}
%%
%%
%%\begin{document}
%%
%%\begin{figure}[h]
%%\centering
%%\scalebox{.5}{\input{r4.tex}}
%%\caption{Monotomic chain dispersion relation}
%%\end{figure}
%%
%%\begin{floatingfigure}{2.9in}
%%%\resizebox{2.5in}{!}{\input{r4}}
%%\scalebox{.5}{\input{r4}}
%%\end{floatingfigure}
%%\noindent The figure to the right provides an illustration of L’H\^o\-pi\-tal’s Rule.  Recall
%%that this rule can be applied when taking the limit as $x\longrightarrow x_a$
%%of a ratio of two functions where the ratio approaches the indeterminate form
%%$\frac{0}{0}$; in the case where both functions are differentiable at $x_a$,
%%the ratio approaches the ratio of their derivatives. In the case illustrated both
%%$\sin(x)$ and $x \longrightarrow0$ as we approach the origin, but the ratio of
%%their derivatives, $\frac{\cos(x)}{1} \longrightarrow 1$. L’H\^opital’s Rule
%%also applies in the case of the indeterminate form $\frac{\infty}{\infty}$.
%%\end{document}
